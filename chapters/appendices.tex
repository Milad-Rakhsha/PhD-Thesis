%!TEX root = ../dissertation.tex
\svnidlong
{$HeadURL: https://subversion.cae.wisc.edu/svn/sbel/Theses/Hammad_PhD/chapters/appendices.tex $}
{$LastChangedDate: 2016-04-30 17:36:33 -0500 (Sat, 30 Apr 2016) $}
{$LastChangedRevision: 9339 $}
{$LastChangedBy: hammad $}
\svnid{$Id: appendices.tex 9339 2016-04-30 22:36:33Z hammad $}
\chapter{Appendices}

\input{chapters/appendix/equivalence}
%\input{chapters/appendix/fem_equivalence}
%  \section{Comparison to classical FEM}
%  In the following we discuss the relationship between the constraint based FEM formulation and the classical force based FEM considering a single tetrahedral element and one simulation step. 

% Start by defining a potential energy function for a system of springs and point masses

% In order to compute the forces generated by the spring we compute its partial derivative

% The derivative represents the direction that the spring forces will act

