%!TEX root = ../dissertation.tex
\chapter{Computational Methods}
\label{chap:NumSolAspects}
\section{Time Integration of Continuum Systems}
and Space Discretization
\subsection{Projection Method}
This algorithm is based on the Helmholtz-Hodge decomposition, which states that any vector field $\vect{u}$ may be decomposed into a  solenoidal ($\vect{u}_{sol}$) and an irrotational ($\vect{u}_{irrot}$) part as follows:
\begin{equation}
\vect{u}=\vect{u}_{irrot}+\vect{u}_{sol},
\end{equation}
where the solenoidal part is divergent-free ($\nabla. \vect{u}_{sol}=0$). The irrotational part is curl-free ($\nabla \times \vect{u}_{irrot=0}$) and can be written as the gradient of a potential/scalar field ($\vect{u}_{irrot}=\nabla \phi$). Interestingly, the irrotational part is a conservative vector field, and the line integral of this field is path-independent.   
By taking divergence of $\vect u$ and using the divergent-free property of $\vect{u}_{sol}$, it can be seen that 
\begin{equation}\label{eq:phi_Poisson}
\nabla . \vect{u}=\nabla .\vect{u}_{irrot}= \nabla^2 \phi,
\end{equation}
and given that $\vect{u}$ is known, the Poisson equation \ref{eq:phi_Poisson} can be solved for $\phi$ which finally results in $\vect{u}_{sol}=\vect{u} -\vect{u}_{irrot}=\vect{u}-\nabla \phi$.

In a similar fashion, we can decompose the force field per unit mass ($\dfrac{D{\vect u}}{Dt}$)  of Equation \ref{eq:NS_IN_Lagrangian} into a solenoidal field ($ \nu \nabla^{2} {\vect u}+{\vect{f}_b}$) and an irrotational field ($-\dfrac{1}{\rho} \nabla p$). The semi-discrete form (discretized only in time) of Eq.~\ref{eq:NS_IN_Lagrangian} is written as follows:
\begin{equation}
\dfrac{(\vect{u}^{n+1}-\vect{u}^n)}{\Delta t}= \dfrac{(\vect{u}^{n+1}-\vect{u}^*)}{\Delta t} +\dfrac{(\vect{u}^{*}-\vect{u})}{\Delta t} = -\dfrac{1}{\rho} \nabla p +\nu \nabla^2\vect{u}^n + \vect{g}
\end{equation}

\subsection{Non-Incremental Projection Method}
Based on the previous discussion we do the following decomposition: 
\begin{align}
\dfrac{(\vect{u}^*-\vect{u}^n)}{\Delta t}=&+\nu \nabla^2\vect{u}^n + \vect{g}=\vect{f}^n \quad x\in \Omega, \qquad \vect{u}^*=0 \quad x\in \partial\Omega\label{eq:predict_HH} \\
\dfrac{(\vect{u}^{n+1}-\vect{u}^*)}{\Delta t}=&-\dfrac{1}{\rho} \nabla p^{n+1}\quad x\in \Omega, \qquad \nabla.\vect{u}^{n+1}=0 \quad \in \partial\Omega \label{eq:correct_HH}.
\end{align}
Eq.~\ref{eq:predict_HH} is the predictor step and it can be used to find the intermediate velocity $\vect{u}^*$:
\begin{equation}\label{eq:u_star}
\vect{u}^*=\vect{f}^n {\Delta t}+\vect{u}^n.
\end{equation}
Moreover, if pressure is known, Eq.~\ref{eq:correct_HH} may be used to find $\vect{u}^{n+1}$ as follows
\begin{equation}\label{eq:u_new}
\vect{u}^{n+1}=-{\Delta t}\dfrac{1}{\rho} \nabla p^{n+1}+\vect{u}^*.
\end{equation}
In order to obtain the pressure equation, using the approach led to Eq.~\ref{eq:phi_Poisson}, we take divergence of the Eq.~\ref{eq:correct_HH} and obtain the Poisson equation for pressure. 
\begin{equation}\label{eq:del_2p_HH}
\dfrac{\nabla.\vect{u}^{n+1}-\nabla.\vect{u}^*}{\Delta t}=-\dfrac{1}{\rho} \nabla^2 p^{n+1}
\end{equation}
Note that from continuity Eq.~\ref{eq:continuity_1} with incompressible flow assumption $\dfrac{D\rho}{Dt}=0$ we can write  $\nabla . \vect{u}^{n+1} =0$ and simplify Eq.~\ref{eq:del_2p_HH} to 
\begin{equation}\label{eq:pressure_1}
\dfrac{1}{\rho} \nabla^2 p^{n+1}=\dfrac{\nabla.\vect{u}^*}{\Delta t}, \quad \nabla p^{n+1}\;.\;\vect{n}|_{\partial \Omega}=0
\end{equation}
Next, Eq.~\ref{eq:u_new} is used to update the velocities. The algorithm described above, \textit{velocity-based projection}, is usually preferred when density variation is small and $\frac{D\rho}{Dt}=0$ is hold. For Lagrangian methods such as SPH, when working with free-surface flows, the \textit{density-based projection} method is considered to be more accurate. In this method continuity equation is used to replace the velocity divergence term $\frac{\nabla.\vect{u}^*}{\Delta t}$ in Eq.~\ref{eq:pressure_1}. The semi-discrete continuity equation is as follows
\begin{equation}\label{eq:cont_desc}
\dfrac{\rho^*-\rho^n}{\Delta t}=-\rho^n \nabla.\vect{u}^*.
\end{equation}
Using the right hand side of \ref{eq:cont_desc}, one may write  Eq.~\ref{eq:pressure_1} as follows:
\begin{equation}\label{eq:pressure_2}
\dfrac{1}{\rho} \nabla^2 p^{n+1}=-\dfrac{1}{\rho^n}\dfrac{\rho^*-\rho^n}{\Delta t^2},
\end{equation}
which takes into consideration the density variation as a source term in the Poisson equation.\\


Another important aspect of the discretization of Eq.~\ref{eq:predict_HH} has to do with the treatment of the viscous forces. This discretization may be chosen to be explicit, implicit or semi-implicit in terms of $\vect{u}^*$  as follows:	
\begin{equation}
\dfrac{(\vect{u}^*-\vect{u}^n)}{\Delta t}=+\nu(\theta \nabla^2\vect{u}^* + (1-\theta)\nabla^2\vect{u}^n)+ \vect{g}=\vect{f}^n\label{eq:CN},
\end{equation}
Eq.~\ref{eq:CN} corresponds to the fully-explicit scheme (Eq~\ref{eq:predict_HH}) for $\theta=0$, and to fully-implicit scheme for $\theta=1$. Crank-Nicolson discretization is obtained via $\theta=0.5$.

\subsection{Incremental Projection Method}
The incremental projection method with the Crank-Nicolson correction applied to the prediction step is as follows:
\begin{align}
\dfrac{(\vect{u}^*-\vect{u}^n)}{\Delta t}=&\frac{\nu}{2}(\nabla^2\vect{u}^* +\nabla^2\vect{u}^n) -\dfrac{1}{\rho} \nabla^2 p+ \vect{g}=\vect{f}^n\label{eq:CN}.\\
\dfrac{(\vect{u}^{n+1}-\vect{u}^*)}{\Delta t}=&-\dfrac{1}{\rho} (\nabla p^{n+1}-\nabla p^{n+1} )=-\dfrac{1}{\rho} (\nabla \phi ) \label{eq:correct_inc},\\
\phi=&p^{n+1}-p^{n}
\end{align}
Herein, the Poisson equation is solved for $\phi$ instead of $p^{n+1}$. In Eulerian grid, the pressure correction is performed via $p^{n+1}=p^{n}+\phi$, whereas in the Lagrangian methods the pressure correction is as follows:
\begin{equation}
p^{n+1}=p^{n}+\phi +\nabla p^{n+1}.(\vect{x}^{n+1}-\vect{x}^{n})
\end{equation}
See \cite{Trask2ndOrder2015,Hosseini2011} for more details about the advantages of the incremental projection method over the non-incremental variation.\\


\section{Space Discretization of Continuum Systems}
\subsection {Smoothed Particle Hydrodynamics}\label{sec:SPH}
We employ SPH for the spatial discretization of Eqs.~\ref{eq:continuity} and \ref{eq:Cauchy_momentum_Eu} to approximate a function $f$ at location $\mathbf r_i$ as \cite{Monaghan2005a}
\begin{equation}\label{eq:SPH_f}
f(\mathbf r_i) \approx  \langle f\rangle_i= \sum_{j \in \support{i}} \frac{m_j}{\rho_j}f(\mathbf r_j)W_{ij} \; ,
\end{equation}
where $\langle f \rangle_i$ indicates the SPH approximation of $f$ at the location of SPH particle, or marker, $i$;  $\support{i}$ represents the collection of SPH particles found in the support domain associated with particle $i$; $\rho_j$ is the density $\rho({\bf r}_j)$ at location ${\bf r}_j$ of particle $j$; $m_j=\rho_j V_j$ and $V_j=(\sum_{k \in {\cal S}(j)} W_{jk})^{-1}$ are the mass and volume associated with marker $j$, respectively; $W_{ij}\equiv W(|\br_i-\br_j|,h)$, where $|{\bf r}|$ is the length of ${\bf r}$. The kernel function $W$ can assume various expressions, e.g., a cubic spline kernel for 3D problems:
\begin{equation} 
\label{eq:kernelExample}
W(|\br|,h) = \frac{5}{{14\pi {h^3}}} \times \left\{ \begin{aligned}
&{(2 - q)^3} - 4{(1 - q)^3}, && 0 \le q < 1 \\ 
&{(2 - q)^3}, && 1 \le q < 2 \\ 
&0, && q \ge 2 \\ 
\end{aligned} \right.,
\end{equation}
where, if the kernel function is located at the origin, $q\equiv | \mathbf{r} | / h$.  The radius of the support domain, $\kappa h$, is proportional to the characteristic length $h$ through the parameter $\kappa$, the latter commonly set to $2$ for the cubic spline kernel.

The standard SPH gradient and Laplacian approximations assume the following expressions, respectively \cite{Monaghan2005a}:
\begin{align}
&\nabla f(\mathbf r_i) \approx \langle \nabla f \rangle_i=\sum_{j \in \support{i}} V_j (f_j-f_i) \nabla_i W_{ij}\label{eq:Standard_G}\\
&\nabla^2 f(\mathbf r_i) \approx \langle \nabla^2 f \rangle_i=2\sum_{j \in \support{i}} V_j ( \mathbf{e}_{ij} \cdot \nabla_i W_{ij}) \frac{f_i-f_j}{|\mathbf{r}_{ij}|} \; , \label{eq:Standard_L}
\end{align}
where $\mathbf{e}_{ij}=\dfrac{\mathbf{r}_{ij}}{|\mathbf{r}_{ij}|}$ and $\nabla_i$ denotes differentiation in space with respect to the coordinates of SPH particle $i$; i.e., 
\begin{equation}
\label{eq:nablaW}
\nabla_i W_{ij} =\left.\frac{\mathbf r_{ij}}{|\mathbf r_{ij}|} \frac{\partial W}{\partial q} \frac{\partial q}{\partial |\mathbf r_{ij}|}\right\vert_{i,j} =  \frac{-15{\mathbf r}_{ij}}{{14\pi {h^5}q}} \times \left\{ \begin{aligned}
&{(2 - q)^2} - 4{(1 - q)^2}, && 0 \le q < 1 \\ 
&(2 - q)^2, && 1 \le q < 2 \\ 
&0, && q \ge 2 \\ 
\end{aligned} \right.\; .
\end{equation}
%For simplicity, in what follows, $\nabla_i W_{ij}$ is replaced by $\nabla W_{ij}$. 

Elaborating on the concept of convergence and accuracy, if a numerical discretization matches the first $m$ terms of the Taylor expansion of the solution, then the numerical approximation
is said to be $(m + 1)^{th}$-order accurate and $\mathcal{C}^m$ consistent. The standard SPH discretizations have $\mathcal{C}^1$  consistency (exact approximation of linear functions) in the interior of the domain provided a regular particle distribution is maintained. If the particle regularity is lost over time, the standard discretization is no longer $\mathcal{C}^1$ consistent and corrections are in order to maintain the consistency order. However, once the kernel function is altered to retain consistency, the SPH discretization will forfeit its symmetry attributes thus losing its conservation trait. Whereas conservative schemes are essential for discretization of the pressure gradient term and pressure driven flows, consistent schemes play a more important role in viscous flows. One side effect of using a consistent discretization is that it requires smaller kernel support \cite{Trask2015,islam2018consistency}. Reducing the support size and thus the number of neighbors is more critical in implicit solvers since linear system fill-in is dictated by the number of SPH particle neighbors. This is the rationale for using a consistent formulation for the implicit SPH method discussed herein, see \S\ref{sec:ISPH}, whenever handling an interior flow scenario.


\noindent The ``consistent'' flavor of the discretization already defined for the conservative case in Eqs.~\ref{eq:Standard_G} and~\ref{eq:Standard_L} assumes the expression \cite{fatehi2011,randles1996}
\begin{align}
&\nabla f(\mathbf r_i) \approx \langle \nabla f \rangle_i=\sum_{j \in \support{i}} V_j (f_j-f_i)\; \mathbf{G}_i\; \nabla_i W_{ij},\label{eq:Consistent_G}\\
&\nabla^2 f(\mathbf r_i) \approx \langle \nabla^2 f \rangle_i=2\sum_{j \in \support{i}}  \left[ \mathbf{L}_i \;:\;  (\mathbf{e}_{ij} \otimes \nabla_i W_{ij}) \right] \Bigg( \frac{f_i-f_j}{|\mathbf{r}_{ij}|}  - \mathbf{e}_{ij} \cdot \nabla f_i\Bigg) V_j,\label{eq:Consistent_L}
\end{align}
where ``$\otimes$" represents the dyadic product of the two vectors; ``$:$" represents the double dot product of two matrices; and $\mathbf{G}_i$ and $\mathbf{L}_i$ are  second-order, symmetric correction tensors. The  $(m,n)$ element of the inverse of $\mathbf{G}_{i}$ is expressed as \cite{Libersky1993,randles1996,fatehi2011}: 
\begin{equation}\label{eq:gradient_Gi}
(\textbf{G}_{i}^{-1})^{mn}=-\sum\limits_j r_{ij}^{m}\nabla_{i,n}W_{ij}V_{j} \;.
\end{equation}
\noindent The $3 \times 3$ matrix $\mathbf{L}_i$ is symmetric and has six unknowns obtained as the solution of a linear system \cite{fatehi2011}. The required six independent equations may be obtained by expanding the following equation for the upper/lower triangular elements of a $3 \times 3 $ matrix, e.g. $m=1,n=1,2,3$,   $m=2,n=2,3$, and $m=3,n=3$.
\begin{equation}\label{eq:delta_mn}
-\delta^{mn}=\sum\limits_j(A_{i}^{kmn}e_{ij}^{k}+r_{ij}^{m}e_{ij}^{n})(L_{i}^{op}e_{ij}^{o}\nabla_{i,p}W_{ij}V_{j}) \; ,
\end{equation}
where $\delta^{mn}$ is the Kronecker symbol, and the elements of the third order tensor $A_{i}$ are obtained as 
\begin{equation}\label{equ:Ai_kmn}
A_{i}^{kmn}=\sum\limits_j r_{ij}^{m} r_{ij}^{n}G_{i}^{kq}\nabla_{i,q}W_{ij}V_{j} \;.
\end{equation}
A detailed account of obtaining the elements $\textbf{L}_i$ is provided in \cite{TR-2016-14}. 

The outcome of the SPH discretization steps described above can be conveniently represented in matrix form. To that end, at the beginning of a time step one computes and stores the discretization matrices $\mathbf{A}^G$ and $\mathbf{A}^L$ that arise from either the standard discretization Eqs.~\ref{eq:Standard_G}-\ref{eq:Standard_L}, or the consistent discretization of  Eqs.~\ref{eq:Consistent_G}-\ref{eq:Consistent_L}. For instance, working with Eq.~\ref{eq:Standard_L}, $\langle \nabla^2 f \rangle_i$ can be expressed as follows:
\begin{align}\renewcommand{\arraystretch}{1.2}
&\langle \nabla^2 f \rangle_i=\mathbf{A}^L_i \mathbf{f},\\
&\mathbf{f}= \begin{bmatrix}
f_1&f_2&\cdots&f_{\nParticles}
\end{bmatrix}^T\\
&\mathbf{A}^L_i= \begin{bmatrix}
\cdots, & 
\smash[b]{\underbrace{\begin{matrix}2\sum_{j \in \support{i}} V_j ( \mathbf{e}_{ij} \cdot \nabla_i W_{ij}) \frac{1}{|\mathbf{r}_{ij}|} \end{matrix}}_{i^{th}\text{ element} }},
&\cdots, &\smash[b]{\underbrace{\begin{matrix}-2 V_j ( \mathbf{e}_{ij} \cdot \nabla_i W_{ij}) \frac{1}{|\mathbf{r}_{ij}|}\end{matrix}}_{j^{th}\text{ element s.t. }{j \in \support{i}} }}, & \cdots
\end{bmatrix}
\end{align}
\\
where the subscript ${\nParticles}$ denotes the number of SPH particles in the domain. Similarly, the gradient of a scalar field $\langle \nabla f \rangle_i$ and divergence of a vector field $\langle \nabla . \mathbf{u} \rangle_i$ may be computed from Eq.~\ref{eq:Standard_G} as follows:
\begin{align}\renewcommand{\arraystretch}{1.2}
&\langle \nabla f \rangle_i=،
\begin{bmatrix} 
\mathbf{A}^{Gx}_i\\
\mathbf{A}^{Gy}_i\\
\mathbf{A}^{Gz}_i\\
\end{bmatrix}\mathbf{f}, \\ 
&\langle \nabla \cdot \mathbf{u} \rangle_i=، 
\mathbf{A}^{Gx}_i \mathbf{u}_x+
\mathbf{A}^{Gy}_i \mathbf{u}_y+
\mathbf{A}^{Gz}_i \mathbf{u}_z, \label{eq:divergence_disc}\\
&\mathbf{f}= \begin{bmatrix}
f_1,&f_2,&\cdots,&f_{\nParticles}
\end{bmatrix}^T, \\
&\mathbf{u}_x= \begin{bmatrix}
(u_x)_{1},&(u_x)_{2},&\cdots,&(u_x)_{\nParticles}
\end{bmatrix}^T,\\
&\mathbf{u}_y= \begin{bmatrix}
(u_y)_{1},&(u_y)_{2},&\cdots,&(u_y)_{\nParticles}
\end{bmatrix}^T,\\
&\mathbf{u}_z= \begin{bmatrix}
(u_z)_{1}, & (u_z)_{2},&\cdots,& (u_z)_{\nParticles}
\end{bmatrix}^T, \label{eq:uxuyuz}
\end{align}
where 
\begin{align}\renewcommand{\arraystretch}{1.2}
\mathbf{A}^{Gx}_i=& \begin{bmatrix}
\cdots, & 
\smash[b]{\begin{matrix}-\sum_{j \in \support{i}} V_j \nabla_{i,1} W_{ij} \end{matrix} },
&\cdots, &\smash[b]{{\begin{matrix}V_j \nabla_{i,1} W_{ij}\end{matrix}}}, & \cdots
\end{bmatrix}\\
\mathbf{A}^{Gy}_i=& \begin{bmatrix}
\cdots, & 
\smash[b]{\begin{matrix}-\sum_{j \in \support{i}} V_j \nabla_{i,2} W_{ij} \end{matrix} },
&\cdots, &\smash[b]{{\begin{matrix}V_j \nabla_{i,2} W_{ij}\end{matrix}}}, & \cdots
\end{bmatrix}\\
\mathbf{A}^{Gz}_i=& \begin{bmatrix}
\cdots, & 
\smash[b]{\begin{matrix}-\sum_{j \in \support{i}} V_j \nabla_{i,3} W_{ij} \end{matrix} },
&\cdots, &\smash[b]{{\begin{matrix}V_j \nabla_{i,3} W_{ij}\end{matrix}}}, & \cdots
\end{bmatrix}\\
\mathbf{A}^{G}_i=& \begin{bmatrix}
\cdots, & 
\smash[b]{\underbrace{\begin{matrix}-\sum_{j \in \support{i}} V_j \nabla_i W_{ij} \end{matrix}}_{i^{th}\text{ element} }} ,
&\cdots, &\smash[b]{\underbrace{\begin{matrix}V_j \nabla_i W_{ij}\end{matrix}}_{j^{th}\text{ element s.t. }{j \in \support{i}} }}, & \cdots
\end{bmatrix}\label{eq:AGMat}
\end{align}
\\
The same approach can be used to obtain the discretization matrices for the consistent discretization of Eqs.~\ref{eq:Consistent_G}-\ref{eq:Consistent_L}. The system level matrices $\mathbf{A}^G$ and $\mathbf{A}^L$ are obtained as

\begin{align}\renewcommand{\arraystretch}{1.2}
&\langle \nabla f \rangle^x= \renewcommand{\arraystretch}{1.2}
\begin{bmatrix}
\langle \nabla f \rangle^x_1\\
\langle \nabla f \rangle^x_2\\
\vdots\\
\langle \nabla f \rangle^x_{\nParticles}
\end{bmatrix}=\mathbf{A}^{Gx}\mathbf{f},\quad
\langle \nabla f \rangle^y= \renewcommand{\arraystretch}{1.2}
\begin{bmatrix}
\langle \nabla f \rangle^y_1\\\langle \nabla f \rangle^y_2\\\vdots\\\langle \nabla f \rangle^y_{\nParticles}
\end{bmatrix}=\mathbf{A}^{Gy}\mathbf{f},\quad
\langle \nabla f \rangle^z= \renewcommand{\arraystretch}{1.2}
\begin{bmatrix}
\langle \nabla f \rangle^z_1\\\langle \nabla f \rangle^z_2\\\vdots\\\langle \nabla f \rangle^z_{\nParticles}
\end{bmatrix}=\mathbf{A}^{Gz}\mathbf{f},\label{eq:grad_op}\\
&\langle \nabla \cdot \mathbf{u} \rangle=  \begin{bmatrix}
\langle \nabla \cdot \mathbf{u}\rangle_1,&\langle \nabla \cdot \mathbf{u} \rangle_2,&\cdots,&\langle \nabla \cdot \mathbf{u} \rangle_{\nParticles}
\end{bmatrix}^T=، 
\mathbf{A}^{Gx} \mathbf{u}_x+
\mathbf{A}^{Gy} \mathbf{u}_y+
\mathbf{A}^{Gz} \mathbf{u}_z, \label{eq:Divergence_discretized}\\
&\mathbf{A}^{Gx}= \renewcommand*{\arraystretch}{1.5}
\begin{bmatrix}\mathbf{A}^{Gx}_1\\\mathbf{A}^{Gx}_2\\\vdots\\\mathbf{A}^{Gx}_{\nParticles}
\end{bmatrix},\quad
\mathbf{A}^{Gy}= \begin{bmatrix}
\mathbf{A}^{Gy}_1\\\mathbf{A}^{Gy}_2\\\vdots\\\mathbf{A}^{Gy}_{\nParticles}
\end{bmatrix},\quad
\mathbf{A}^{Gz}= \begin{bmatrix}
\mathbf{A}^{Gz}_1\\\mathbf{A}^{Gz}_2\\\vdots\\\mathbf{A}^{Gz}_{\nParticles}
\end{bmatrix}\\
&\langle \nabla^2 f \rangle=\mathbf{A}^L\mathbf{f},\label{eq:laplace_op}\\
&\langle \nabla^2 f \rangle= \begin{bmatrix}
\langle \nabla^2 f \rangle_1,&\langle \nabla^2 f \rangle_2,&\cdots,&\langle \nabla^2 f \rangle_{\nParticles}
\end{bmatrix}^T,\\
&\mathbf{A}^L=  \renewcommand*{\arraystretch}{1.5}
\begin{bmatrix}
\mathbf{A}^L_1\\\mathbf{A}^L_2\\\vdots\\\mathbf{A}^L_{\nParticles}
\end{bmatrix} \; .
\end{align}
This allows for the space discretization of the incompressible Navier-Stokes equations in the $x$, $y$, and $z$ directions, per Eq.~(\ref{eq:NS_IN_Lagrangian}),
\begin{align}
\begin{cases}
\frac{d{\bf u}_x}{dt} & \approx -\frac{1}{\rho} \mathbf{A}^{Gx}\mathbf{p} + \nu \mathbf{A}^L{\bf u}_x+{ f}_x^b\\
\frac{d{\bf u}_y}{dt} & \approx -\frac{1}{\rho} \mathbf{A}^{Gy}\mathbf{p} + \nu \mathbf{A}^L{\bf u}_y+{ f}_y^b\\
\frac{d{\bf u}_z}{dt} & \approx -\frac{1}{\rho} \mathbf{A}^{Gz}\mathbf{p} + \nu \mathbf{A}^L{\bf u}_z+{ f}_z^b
\end{cases}\label{eq:NS_discretized},
\end{align}
where 
\begin{align}
&\mathbf{p}= \begin{bmatrix}
p_{1},&p_{2},&\cdots,&p_{\nParticles}
\end{bmatrix}^T,\label{eq:pres_vec}
\end{align} 
is the vector of pressures and 
and ${\bf u}_x$, ${\bf u}_y$, and ${\bf u}_z$ were defined in Eq.~\ref{eq:uxuyuz}.


\subsubsection{Particle Shifting}\label{subsec:Par_Shift}  %2-3
The advection of SPH particles can lead to scenarios characterized by high particle disorder and/or regions with high particle depletion/plenitude. Maintaining the accuracy and stability of the SPH method under these circumstances requires mitigating measures, one being particle shifting. The latter calls for retiring a particle only to introduce it back in a consistent fashion, slightly away from the previous location and the streamlines in order to improve the uniformity of the SPH particle distribution. Particle shifting in conjunction with an ISPH-type implementation was used in \cite{Xu2009Accuracy} and subsequently reported to be effective in \cite{Trask2015}. It has also been used in conjunction with a WCSPH-class implementation in \cite{Shadloo2012Robust}. The shifting vector is computed for particle $i$ as
\begin{equation*}
\delta \textbf{r}_{i} = \beta r_0^2 u_\mathrm{max} \Delta t\sum\limits_{j \in \support{i}}  \frac{\textbf{r}_{ij}}{r_{ij}^3} \; ,
\end{equation*}
where $r_0=\frac{1}{N_i}\sum\limits_j  {r_{ij}}$, $u_\mathrm{max}$ is the maximum velocity in the domain, $\Delta t$ is the integration time step, and $\beta$ is an adjustable dimensionless parameter determining the magnitude of the shifting vector. At the end of each time step the position of particle $i$ is shifted by $\textbf{x}_{i}^{new} = \textbf{x}_{i} + \delta \textbf{r}_{i}$. Accordingly, the field variable $\rho_{i}$, $p_{i}$, and $\textbf{v}_{i}$ are updated as 
\begin{equation*}
p_{i}^{new} = p_{i} + \nabla p_{i}\cdot \delta \textbf{r}_{i} \; , \quad
\rho_{i}^{new} = \rho_{i} + \nabla \rho_{i}\cdot \delta \textbf{r}_{i} \;, \quad \mbox{and} \quad
\mathbf{u}_{i}^{new} = \mathbf{u}_{i} + \nabla \mathbf{u}_{i}\cdot \delta \textbf{r}_{i} \; .
\end{equation*}

\subsection{WCSPH}
The cornerstone of WCSPH is its use of a state equation to obtain the pressure from density -- at marker $i$,
\begin{equation}
\label{eq:State_eq}
p_i = (k| {\bf u}|_{max})^2\big(\frac{\rho_i}{\rho_0}-1\big)+p_0 \; ,
\end{equation}
where $k|{\bf u}|_{max}$ is a sound speed proxy; $|{\bf u}|_{max}$ is the magnitude of the maximum velocity in the domain; $k=10$ is an empirical scaling factor; and $\rho_0$ and $p_0$ are reference values. The density update may be obtained from the time integration of the continuity equation Eq.~\ref{eq:continutiy} using the space discretization of the velocity divergence from Eq.~(\ref{eq:Divergence_discretized}) 
\begin{equation}
\label{eq:rho2}
\frac{\mathbf{\rho}^{n+1}-\mathbf{\rho}^{n}}{\Delta t}=-<\mathbf{\rho}^n \cdot ({\mathbf{A}^{Gx} \mathbf{u}_x+\mathbf{A}^{Gy} \mathbf{u}_y+\mathbf{A}^{Gz}\mathbf{u}_z})> \;,
\end{equation}
where $\mathbf{\rho}^{n}\equiv\begin{bmatrix}
\rho_0,& \rho_1,& \cdots,& \rho_{\nParticles}\end{bmatrix}^T$ is the system level vector of markers' density at the \textit{current} time step, and $\mathbf{c}=<\mathbf{a} \cdot \mathbf{b}>$ indicates the \textit{element-wise} product of vector $\mathbf{a}$ and $\mathbf{b}$.
The time integration used is of predictor-corrector type:
\begin{align*}
\text{predictor stage:} \quad
\begin{cases}
{\bar{\mathbf{u}}}^{n+1/2}=\mathbf{u}^n+ \frac{\Delta t}{2} \mathbf{a}^n  \\ 
{\bar{\mathbf{x}}}^{n+1/2}=\mathbf{x}^n+ \frac{\Delta t}{2} \mathbf{u}^n
\end{cases} ,
\end{align*}
and
\begin{align*} 
\text{corrector stage:} \quad
\begin{cases}
{\mathbf{u}}^{n+1/2}=\mathbf{u}^n+ \frac{\Delta t}{2} {\bar{\mathbf{a}}}^{n+1/2} \\ 
{\mathbf{x}}^{n+1/2}=\mathbf{x}^n+ \frac{\Delta t}{2} \mathbf{u}^{n+1/2}
\end{cases} .
\end{align*}
Finally,
\begin{align*} 
\text{update stage:} \quad
\begin{cases}
{\mathbf{u}}^{n+1}=2\mathbf{u}^{n+1/2}-\mathbf{u}^n  \\ 
{\mathbf{x}}^{n+1}=2\mathbf{x}^{n+1/2}-\mathbf{x}^n
\end{cases} .
\end{align*}
Above, $\mathbf{a}^n$ is the system-level vector of accelerations at time step $n$; its components in the $x$, $y$, $z$ directions are obtained from the space discretization of the Navier-Stokes as   
\begin{align*}
\begin{cases}
(\mathbf{a}^x)^n=  -\frac{1}{\rho} (\mathbf{A}^{Gx})^n\mathbf{p}^n + \nu (\mathbf{A}^L)^n{\bf u}^n_x+{ f}_x^b\\
(\mathbf{a}^y)^n=  -\frac{1}{\rho} (\mathbf{A}^{Gy})^n\mathbf{p}^n + \nu (\mathbf{A}^L)^n{\bf u}^n_y+{ f}_y^b\\
(\mathbf{a}^z)^n=  -\frac{1}{\rho} (\mathbf{A}^{Gz})^n\mathbf{p}^n + \nu (\mathbf{A}^L)^n{\bf u}^n_z+{ f}_z^b
\end{cases} .
\end{align*}
Likewise, ${\bar{\mathbf{a}}}^{n+1/2}$ in the corrector stage is obtained using ${\bar{\mathbf{x}}}^{n+1/2}$, ${\bar{\mathbf{u}}}^{n+1/2}$, and the associated discrete representation of the gradient and Laplacian operators, i.e. $({\bar{\mathbf{A}}}^G)^{n+1/2}$, and $({\bar{\mathbf{A}}}^L)^{n+1/2}$. As far as the time step $\Delta t$ is concerned, its size is constrained on numerical stability grounds by the following condition \cite{liu2003smoothed}:
\begin{align}\label{eq:timestep}
\Delta t \leq \text{min} \begin{Bmatrix}
0.25 \dfrac{h}{k| {\bf v}|_{max}},& 0.125 \dfrac{h^2}{\nu}, & 0.25 \sqrt{\dfrac{h}{|\mathbf{f}_b|}}\end{Bmatrix} \; .
\end{align}
Above, the first term corresponds to the CFL condition and is the place where the ``numerical stiffness'' of the state equation in Eq.~(\ref{eq:State_eq}) comes into play. Specifically, the higher the $k$ value, the lower the amount of allowed compressibility, and, at the same time, the smaller the time step. The second restriction in Eq.~(\ref{eq:timestep}) appears due to the explicit treatment of the viscous term and restricts the time step by a factor that is inversely proportional to the viscosity -- the higher the viscosity, the lower the time step. More importantly, the second restriction is also proportional to $h^2$, which significantly and adversely impacts the time step when a finer particle distribution is employed. The last restriction is due to the explicit treatment of the external (body) forces. Ultimately, these relatively stringent bounds on the time step $\Delta t$ prompted the search for alternative SPH-based approaches; e.g., ISPH and KCSPH.

\subsection{ISPH}
\label{sec:ISPH}
ISPH alleviates the time step constraints hindering WCSPH at the price of solving a linear system of equations at each time step. It draws on the Helmholtz-Hodge decomposition and Chorin's projection method \cite{chorin1968numerical} to integrate the continuity and the Navier-Stokes equation as:
\begin{align}
\text{prediction:} \quad &\begin{cases}\label{eq:predict} 
\dfrac{(\mathbf{u}^*-\mathbf{u}^n)}{\Delta t}=\frac{\nu}{2}(\nabla^2\mathbf{u}^* +\nabla^2\mathbf{u}^n) + \mathbf{f}^b\quad x\in \Omega,\\
\mathbf{u}^*=0 \quad x\in \partial\Omega.\\
\end{cases}  \\
\text{correction:} \quad &\begin{cases}\label{eq:correct} 
\dfrac{(\mathbf{u}^{n+1}-\mathbf{u}^*)}{\Delta t}=-\dfrac{1}{\rho} \nabla p^{n+1}\quad x\in \Omega, \\
\nabla.\mathbf{u}^{n+1}=0
\end{cases}  
\end{align}
Equation~\ref{eq:predict} is the predictor step used to find the intermediate velocity $\mathbf{u}^*$. Taking divergence of the Eq.~\ref{eq:correct}, the Poisson equation for pressure may be obtained as follows: 
\begin{equation}\label{eq:del_2p}
\dfrac{\nabla.\mathbf{u}^{n+1}-\nabla.\mathbf{u}^*}{\Delta t}=-\dfrac{1}{\rho} \nabla^2 p^{n+1}
\end{equation}
The incompressible flow assumption, $\nabla . \mathbf{u}^{n+1} =0$, can be used to simplify Eq.~\ref{eq:del_2p} to 
\begin{equation}\label{eq:pressure}
\begin{cases}
\dfrac{1}{\rho} \nabla^2 p^{n+1}=\dfrac{\nabla.\mathbf{u}^*}{\Delta t}\\
\nabla p^{n+1}\;.\;\mathbf{n}|_{\partial \Omega}=0
\end{cases} .
\end{equation}
Once the above Poisson problem is solved for pressure, Eq.~\ref{eq:correct} may be used to find $\mathbf{u}^{n+1}$ as
\begin{equation*}
\mathbf{u}^{n+1}=-{\Delta t}\dfrac{1}{\rho} \nabla p^{n+1}+\mathbf{u}^* \;.
\end{equation*}
The algorithm described above is known as the \textit{velocity-based projection} version. However, when working with free-surface flows, it is beneficial to use a \textit{density-based projection} method \cite{asai2012stabilized} to account for the density variation experienced by the SPH particles in the proximity of the free surface. The continuity equation is then used to replace the velocity divergence term $\frac{\nabla.\mathbf{u}^*}{\Delta t}$ in Eq.~\ref{eq:pressure}. The semi-discrete continuity equation corresponding to the prediction state is:
\begin{equation}\label{eq:cont_desc_ISPH}
\dfrac{\rho^*-\rho^n}{\Delta t}=-\rho^n \nabla.\mathbf{u}^*.
\end{equation}
Using the right hand side of Eq.~\ref{eq:cont_desc_ISPH}, one may write  Eq.~\ref{eq:pressure} as
\begin{equation}\label{eq:pressure_density}
\begin{cases}
\dfrac{1}{\rho} \nabla^2 p^{n+1}=-\dfrac{1}{\rho^n}\dfrac{\rho^*-\rho^n}{\Delta t^2}\\
\nabla p^{n+1}\;.\;\mathbf{n}|_{\partial \Omega}=0
\end{cases},
\end{equation}
which takes into consideration the density variation as a source term in the Poisson equation. Following an approach similar to the one introduced in \cite{asai2012stabilized}, we use a stabilization of the source term in the Poisson equation to linearly combine the right-hand sides of Eqs.~\ref{eq:pressure} and \ref{eq:pressure_density}:
\begin{equation*}
\text{Pressure equation:} \quad
\begin{cases}
\dfrac{1}{\rho} \nabla^2 p^{n+1}=\alpha\dfrac{1}{\rho^n}\dfrac{\rho^n-\rho^*}{\Delta t^2} + (1-\alpha) \dfrac{\nabla.\mathbf{u}^*}{\Delta t}\\
\nabla p^{n+1}\;.\;\mathbf{n}|_{\partial \Omega}=0
\end{cases}.
\end{equation*}
This stabilization technique turned out to be critical in simulations where the difference between the density and the rest density ($\rho_0$) is large.

The above time-discretized equations may be combined with the space-discretization of Eqs.~\ref{eq:grad_op} and \ref{eq:laplace_op}. The space-time discretized ISPH equations are as follows:
\begin{align}
&\begin{cases}\label{eq:predict_disc} 
\Big(\frac{1}{\Delta t} \mathbf{I}-\frac{\nu}{2} \mathbf{A}^L\Big)\; \mathbf{u}^*_x=
\Big(\frac{1}{\Delta t} \mathbf{I}+\frac{\nu}{2} \mathbf{A}^L\Big) \; \mathbf{u}^n_x+ {f}^b_x \quad \text{particles}\in \Omega,\\
\Big(\frac{1}{\Delta t} \mathbf{I}-\frac{\nu}{2} \mathbf{A}^L\Big)\; \mathbf{u}^*_y=
\Big(\frac{1}{\Delta t} \mathbf{I}+\frac{\nu}{2} \mathbf{A}^L\Big) \; \mathbf{u}^n_y+ {f}^b_y \quad \text{particles}\in \Omega,\\
\Big(\frac{1}{\Delta t} \mathbf{I}-\frac{\nu}{2} \mathbf{A}^L\Big)\; \mathbf{u}^*_z=
\Big(\frac{1}{\Delta t} \mathbf{I}+\frac{\nu}{2} \mathbf{A}^L\Big) \; \mathbf{u}^n_z+ {f}^b_z \quad \text{particles} \in \Omega,\\
\mathbf{u}^*=0 \quad \text{on } \partial\Omega.
\end{cases}
\end{align}

\begin{align}
&\begin{cases}\label{eq:press_disc} 
\dfrac{1}{\rho} \mathbf{A}^L \mathbf{p}^{n+1}=\alpha\dfrac{1}{\rho^n}\dfrac{\mathbf{\rho}^n-\mathbf{\rho}^*}{\Delta t^2} + (1-\alpha) \dfrac{\mathbf{A}^{Gx} \mathbf{u}_x^*+\mathbf{A}^{Gy} \mathbf{u}_y^*+\mathbf{A}^{Gz}\mathbf{u}_z^*}{\Delta t},\\
\nabla p^{n+1}\;.\;\mathbf{n}|_{\partial \Omega}=0
\end{cases}
\end{align}

\begin{align}
&\begin{cases}\label{eq:correct_disc} 
\dfrac{(\mathbf{u}_x^{n+1}-\mathbf{u}_x^*)}{\Delta t}=-\dfrac{1}{\rho} \mathbf{A}^{Gx} \mathbf{p}^{n+1} \\
\dfrac{(\mathbf{u}_y^{n+1}-\mathbf{u}_y^*)}{\Delta t}=-\dfrac{1}{\rho} \mathbf{A}^{Gy} \mathbf{p}^{n+1}\\
\dfrac{(\mathbf{u}_z^{n+1}-\mathbf{u}_z^*)}{\Delta t}=-\dfrac{1}{\rho} \mathbf{A}^{Gz} \mathbf{p}^{n+1}
\end{cases}
\end{align}

Several key observations pertaining the ISPH method implemented are summarized as follows:
\begin{itemize}
	\item The Crank-Nicolson discretization of the viscous term in Eq.~\ref{eq:predict} leads to a non-diagonal coefficient matrix $(\frac{1}{\Delta t} \mathbf{I}-\frac{\nu}{2} \mathbf{A}^L)$ in Eq.~\ref{eq:predict_disc}. Had one chosen to treat the viscous term in Eq.~\ref{eq:predict} explicitly; i.e., $\nu \nabla^2\mathbf{u}^n$ instead of $\frac{\nu}{2}(\nabla^2\mathbf{u}^* +\nabla^2\mathbf{u}^n)$, the coefficient matrix of the linear system in Eq.~\ref{eq:predict_disc} would have become $\frac{1}{\Delta t} \mathbf{I}$; i.e., a diagonal matrix. Yet this choice that makes the linear solve trivial, would constrain the time step $\Delta t$ owing to the explicit treatment of the viscous term, which imposes the time-step restriction of $\Delta t < 0.125 \frac{h^2}{\nu}$, see Eq.~\ref{eq:timestep}. 
	
	\item The $x$, $y$, and $z$ directions in Eq.~\ref{eq:predict_disc} use the same coefficient matrix, which simplifies the implementation. 
	
	\item A modification that proved particularly useful at small $\Delta t$ pertains a scaling of the pressure by a factor  $\Delta t^2$ in the Poisson equation, and by $1/\Delta t^2$ in the correction step Eq.~\ref{eq:correct_disc}. In other words, we compute $p\Delta t^2$ when solving the Poisson equation, and subsequently scale the pressure in the correction step Eq.~\ref{eq:correct_disc} by a factor of $1/\Delta t^2$.
\end{itemize} 
\subsection{IISPH}
\label{subsec:IISPH}
In the IISPH method of choice \cite{ihmsen2014implicit}, the continuity equation, Eq.~(\ref{eq:Continutiy}), is discretized in time and space using forward Euler and the approximation in Eq.~(\ref{eq:F2}), respectively, to yield
%%%%%%%%%%%%%%%%%%%%%%%
\begin{equation}\label{eq:rhocontinuity}
\frac{\rho_i(t+\Delta t)-\rho_i(t)}{\Delta t}=\sum_{j \in {\cal S}(i)} m_j \mathbf v_{ij}(t+\Delta t). \nabla W_{ij}\;,
\end{equation}
%%%%%%%%%%%%%%%%%%%%%%%
where $\Delta t$ is the time step. The Navier-Stokes equation, Eq.~(\ref{eq:NS}), is discretized using Chorin's two-step projection method \cite{chorin1968numerical}. First, an intermediate velocity, denoted below by $\mathbf v_{ij,l}^{np}$, is obtained using the current state information to evaluate the velocities and the right-hand-side terms of Eq.~(\ref{eq:NS_IN_}) except the pressure gradient term $\nabla p/\rho$. Finally, new velocities are obtained by correcting the intermediate velocity with pressure information at the new time step. In IISPH, the pressure at the new time step is obtained by solving a Poisson equation. The details are as follows. The solution methodology starts off by partitioning the forces in the right-hand-side of the momentum equations:
\begin{equation}\label{eq:continuityDes}
m_i \frac{\mathbf v_i(t_l+\Delta t)-\mathbf v_i(t_l)}{\Delta t}={\bf f}\;^p_i(t_l+\Delta t)+{\bf f}\;^{np}_i(t_l)\;,
\end{equation}
%%%%%%%%%%%%%%%%%%%%%%%
where, in the light of Eq.~(\ref{eq:NS}), ${\bf f}_i^{\:p}$ and ${\bf f}_i^{\:np}$ represent the pressure-related and non-pressure-related forces acting on particle $i$, respectively. After simple manipulations,
\begin{equation}\label{eq:seconLaw}
\mathbf v_{i,l+1}=\bigg(\mathbf v_{i,l}+ \frac{{\bf f}\;^{np}_{i,l}}{ m_i}\Delta t\bigg)+ \frac{{\bf f}\;^{p}_{i,l+1}}{ m_i}\Delta t=\mathbf v^{np}_{i,l}+\mathbf v^{p}_{i,l+1}\;,
\end{equation}
%%%%%%%%%%%%%%%%%%%%%%% 
where $\mathbf v^{np}_{i,l} \equiv \mathbf v_{i,l}+ {{\bf f}\;^{np}_i}/{ m_i}\Delta t$ is the intermediate velocity, $\mathbf v^{p}_{i,l+1} \equiv {{\bf f}\;^{p}_{i,l+1}}\Delta t/{ m_i}$ is the velocity correction, and subscripts $l$ and $l+1$ represent the discretized values at time $t_l$ and $t_l+\Delta t$, respectively. 

Unlike the original Chorin projection method and other incompressible SPH solutions, which obtain the pressure equation by imposing the divergence-free velocity condition, i.e.  $\nabla.\bv= 0$, the pressure equation in IISPH is obtained by imposing $d\rho/dt=0$. The choice of density-invariance condition leads to a more uniform particle distribution and a lower density error whereas a divergence-free velocity condition results in a more accurate pressure distribution \cite{asai2012stabilized}. Hence, in Eq.~(\ref{eq:rhocontinuity}), the density is also divided into the pressure and non-pressure parts. Subsequently, the intermediate density resulting from the intermediate velocity $\mathbf v_{ij,l}^{np}$ is expressed as
%%%%%%%%%%%%%%%%%%%%%%%
%%%%%%%%%%%%%%%%%%%%%%%
\begin{equation}\label{eq:rhonp}
\rho_{i,l}^{np}=\rho_{i,l}+\Delta t \sum_{j \in {\cal S}(i)} m_j \mathbf v_{ij,l}^{np}\nabla W_{ij}.
\end{equation}
%%%%%%%%%%%%%%%%%%%%%%%
Subtracting the above equation from Eq.~(\ref{eq:rhocontinuity}) yields
%%%%%%%%%%%%%%%%%%%%%%%
\begin{equation}\label{eq:continuity3}
\rho_{i,l+1}- \rho_{i,l}^{np}=\Delta t \sum_{j \in {\cal S}(i)} m_j (\mathbf v_{ij,l+1}-\mathbf v_{ij,l}^{np})\nabla W_{ij} \; .
\end{equation}
%%%%%%%%%%%%%%%%%%%%%%%
Incompressibility is enforced by setting $\rho_{i,l+1}= \rho _0 $. Using Eq.~(\ref{eq:seconLaw}) and the definition of $\mathbf v_{ij,l+1}^{p}$, the above equation, after further manipulations, yields the pressure equation that is formulated at the location of each particle $i$ as
%%%%%%%%%%%%%%%%%%%%%%%
\begin{equation}\label{eq:continuity4}
\rho _0- \rho_{i,l}^{np}=\Delta t \sum_{j \in {\cal S}(i)} m_j \mathbf v_{ij,l+1}^{p}\nabla W_{ij}
= \Delta t^2 \sum_{j \in {\cal S}(i)} m_j \Bigg(\frac{{\bf f}\;^{p}_{i,l+1}}{ m_i}-\frac{{\bf f}\;^{p}_{j,l+1}}{ m_j}\Bigg)\nabla W_{ij} \; ,
\end{equation}
%%%%%%%%%%%%%%%%%%%%%%%
where the conservative pressure forces may be obtained from \cite{ihmsen2014implicit}
\begin{equation}\label{eq:Fi}
\begin{split}
\frac{\Delta t^2}{m_i}  {\bf f}\;^{p}_{i,l+1}=& -\Delta t^2 \sum_{j \in {\cal S}(i)} m_j \bigg(\frac{p_{i,l+1}}{\rho_{i,l}^2}+\frac{p_{j,l+1}}{\rho_{j,l}^2}\bigg) \nabla W_{ij} \\=& \underbrace{\Bigg( -\Delta t^2    \sum_{j \in {\cal S}(i)}  \frac{m_j}{\rho_i^2} \nabla W_{ij}\Bigg)}_{\text{ \large $\mathbf{d}_{ii}$}  } p_{i,l+1} + \sum_{j \in {\cal S}(i)} \underbrace{ \bigg( -\Delta t^2 \frac{m_j}{\rho_j^2} \nabla W_{ij}\bigg)}_{\text{ \large $\mathbf{d}_{ij}$}}  p_{j,l+1}\\=& \mathbf{d}_{ii} p_{i,l+1}+\sum_{j \in {\cal S}(i)} \mathbf{d}_{ij} p_{j,l+1} \; ,
\end{split} 
\end{equation}
and the viscous force from \cite{pazouki2014}
\begin{equation}\label{eq:viscosity}
{\bf f}\;^{np}_{i,l}= m_i \sum_{j \in {\cal S}(i)} m_j \Pi _{ij}+{\bf f}_b\; .
\end{equation}
Herein, ${\bf f}_b$ represents the body force density and  $\Pi _{ij} \equiv -\dfrac{(\mu_i+\mu_j)\mathbf x_{ij}\nabla W_{ij} }{\overline{\rho}^2_{ij}(\mathbf x^2_{ij}+\epsilon \overline{h}^2_{ij})}\mathbf v_{ij,l}$,  $ \overline{h}_{ij}= (h_i+ h_j)/2$ and $ \overline{\rho}_{ij}= (\rho_i+ \rho_j)/2$.  Finally, by substituting Eqs.~(\ref{eq:rhonp}) and (\ref{eq:Fi}) into (\ref{eq:continuity4}), a linear system of equations, with pressure as the unknown, is obtained as follows:

\begin{equation}\label{eq:Pressure}
\begin{split}
\rho _0- \rho_{i,l}^{np}= & \sum_{j \in {\cal S}(i)} m_j \Bigg( \mathbf{d}_{ii} p_{i,l+1}+\sum_{t \in {\cal S}(i)} \mathbf{d}_{it} p_{t,l+1} + \mathbf{d}_{jj} p_{j,l+1}+\sum_{k \in {\cal S}(j)} \mathbf{d}_{jk} p_{j,l+1}\Bigg) .\nabla W_{ij}\\
=&p_{i,l+1}\underbrace{ \Bigg(  \sum_{j \in {\cal S}(i)} m_j ( \mathbf{d}_{ii}-\mathbf{d}_{ji}) . \nabla W_{ij}\Bigg ) }_{\text{\large $a_{ii}$}} \\ +& \sum_{j \in {\cal S}(i)} m_j\Bigg( \sum_{t \in {\cal S}(i)} \mathbf{d}_{it} p_{t,l+1}  + \mathbf{d}_{jj} p_{j,l+1}+\sum_{k \in {\cal S}(j), k\neq i} \mathbf{d}_{jk} p_{j,l+1}\Bigg). \nabla W_{ij}\\
=&p_{i,l+1} \; a_{ii} +  \sum_{t \in {\cal S}(i)} p_{t,l+1} \big(\mathbf{d}_{it} \;. \sum_{j \in {\cal S}(i)} m_j \nabla W_{ij}\big) \\ +& \sum_{j \in {\cal S}(i)} p_{j,l+1}  \; m_j \; \mathbf{d}_{jj}. \nabla W_{ij} + \sum_{j \in {\cal S}(i)} m_j  \sum_{k \in {\cal S}(j), k\neq i} \mathbf{d}_{jk} p_{j,l+1}. \nabla W_{ij}.
\end{split} 
\end{equation}
Equation (\ref{eq:Pressure}) represents the pressure equation in the form of a linear system of equations, i.e. $\bA {\bf p}={\bf b}$, where ${\bf p} = \left[ p_1, p_2, p_3, \ldots, p_{N_f} \right]^T$, ${\bf b} = \left[ \rho_0-\rho_{1,l}^{np}, \rho_0-\rho_{2,l}^{np}, \rho_0-\rho_{3,l}^{np},\ldots, \rho_0-\rho_{N_f}^{np}\right]^T$, and $N_f$ denotes the total number of SPH markers.  Similarly, the elements of the sparse matrix $\bA$ can be obtained from Eq. (\ref{eq:Pressure}). 
%
Once the pressure values are available, Eqs.~(\ref{eq:seconLaw}) and~(\ref{eq:Fi}) are used to compute the velocity at the new time step $t_{l+1}$; subsequently, positions are updated as $\mathbf{x}_{i,l+1}=\mathbf{x}_{i,l}+ \Delta t \; \mathbf{v}_{i,l+1}$. 

Unlike the weakly compressible SPH (WCSPH) algorithm invoked in the usual SPH solution \cite{Monaghan2005a}, the methodology described does not use a state equation. The highly oscillatory nature of the solution (induced by the state equation) and the use of explicit integration constrain WCSPH to small integration steps. Conveniently, the numerical method discussed avoids the introduction of high frequency oscillations, which are a numerical artifact, and can advance the solution at large integration time steps. Nevertheless, typical stability criteria still hold. Herein, an adaptive time-stepping scheme is considered in which the time step $\Delta t$ is constrained by the Courant\-Friedrichs\-Lewy (CFL) condition, i.e. $C= v \Delta t / h < C_{max}$, the magnitude of volumetric forces $\left | {\bf f}_b \right |$, and the viscous dissipation \cite{Monaghan1992,Morris1997}:
\begin{equation}
\label{eq:delta_t}
\Delta t \leq \min \{ 
\frac{C_{max} h}{v_{max}},~ 
0.25 \min_i(\frac{h}{\left | {\bf f}_b \right |_i})^{\frac{1}{2}},~
0.125 \frac{h^{2}}{\nu}
\} \; ,
\end{equation}
where subscript $i$ denotes the particle $i$ and $C_{max}$ depends on the integration scheme.  In the present study, $C_{max}<0.1$ was found to be sufficient for the stability of the numerical solution. The solution methodology is outlined in Algorithm \ref{al:A_Matrix}. 
\begin{algorithm}
	\caption{Fluid Dynamics via IISPH}
	{\fontsize{10}{10}\selectfont
		\begin{algorithmic}[1]
			\FORALL{ particles $i$} \STATE {Compute $\rho_{i,l}=\sum_{j \in {\cal S}(i)} m_j  W_{ij}$ } \ENDFOR
			\STATE	Synchronize
			\FORALL{ particles $i$} \STATE {Calculate $\mathbf v^{np}_{i,l}$ \AND $\mathbf{d}_{ii}$} \ENDFOR
			\STATE	Synchronize
			\FORALL{ particles $i$} 
			\STATE Calculate $\rho_{i,l}^{np}$, $a_{ii}$, \AND   $\mathbf{c}_{ii}=\sum_{j \in {\cal S}(i)} m_j \nabla W_{ij}$
			\ENDFOR
			\STATE	Synchronize
			
			\FORALL{ fluid particles $i$} 
			\STATE {Set ${\bf A}(i,i)=a_{ii}$}
			\IF{$\rho_{i,l}>0.99\rho_0$}
			\STATE {Set ${\bf b}(i)=\rho _0- \rho_{i,l}^{np}$}
			\FORALL{ particles ${j \in {\cal S}(i)}$}
			\STATE  ${\bf A}(i,j)+=m_j  \bigg( \mathbf{d}_{ij}.\mathbf{c}_{ii} + \mathbf{d}_{jj}. \nabla W_{ij} \bigg)$
			\FORALL{ particles ${k \in {\cal S}(j)}$}
			\STATE  ${\bf A}(i,k)+=m_j  \bigg(  \mathbf{d}_{jk}. \nabla W_{ij} \bigg)$
			\ENDFOR
			\ENDFOR
			\ENDIF
			\ENDFOR
			
			\FORALL{ BCE particles $i$} 
			\STATE Compute the prescribed wall velocity ($\mathbf{v}_{i,w}$) and acceleration ($\mathbf{a}_{i,w}$) at the position of marker $i$ from the motion of the associated deformable or rigid body. For rigid bodies they can be obtained from Eq.~(\ref{eq:rigid_body_info}), and for deformable bodies from shape functions of Eqs.~(\ref{eq:ANCF_Beam_Shapefunctions}) and (\ref{eq:ANCF_Shell_Shapefunctions}) along with the associated nodal generalized velocity and acceleration of the  corresponding element.
			
			\STATE The velocities of the BCE markers required for the no-slip conditions are obtained  from Eq.~(\ref{eq:vBCE_Adami})
			\STATE  Set $\mathbf{v}_{dummy}=0$, $den=0$ and  
			\FORALL{fluid particles ${j \in {\cal S}(i)}$}
			\STATE ${\bf b}(i)+=\rho_j \bigg(\left( \mathbf{g} - \mathbf{a}_{j,w} \right) \cdot \mathbf{r}_{ij}\bigg) W_{ij}$
			\STATE Set ${\bf A}(i,j)=-W_{ij}$
			\STATE $den+=W_{ij}$ \AND $\mathbf{v}_{dummy}+=\mathbf{v}_{j,w} W_{ij}$
			\ENDFOR
			
			\IF{$den <\epsilon$}
			\STATE $\mathbf{v}_{i}=2\mathbf{v}_{i,w}$
			\STATE ${\bf A}(i,i)=a_{ii}$ \AND ${\bf b}(i)=0$
			\ELSE
			\STATE $\mathbf{v}_{i}=2\mathbf{v}_{i,w}-\mathbf{v}_{dummy}/den$
			\STATE ${\bf A}(i,i)=den$
			\STATE Scale all the elements of row $i$ of ${\bf A}$ and ${\bf b}(i)$ by $a_{ii}/den$ 
			\ENDIF
			\ENDFOR		
			\STATE Solve ${\bf A}{\bf p}={\bf b}$ using BiCGStab or GMRES
			\FORALL{ fluid particles $i$} 
			\STATE Correct velocities from $\mathbf{v}_{i,l+1}=\mathbf{v}_{i,l}^{np}+ \dfrac{{\bf f}\;^{p}_{i,l+1}}{ m_i}\Delta t$ via Eq.~(\ref{eq:Fi})
			\STATE Update the state by $\mathbf{x}_{i,l+1}=\mathbf{x}_{i,l}+ \Delta t \; \mathbf{v}_{i,l+1}$
			\ENDFOR
		\end{algorithmic}\label{al:A_Matrix}
	}
\end{algorithm}
\subsection{KCSPH}
\label{sec:KCSPH}
KCSPH approaches are relatively new and unconventional. They draw on a thermodynamically consistent SPH discretization \cite{espanol_smoothed_2003} that leads to an index 3 set of differential algebraic equations (see Section~\ref{sec:RigidBody} describing the motion of the fluid markers. The SPH particles can be regarded as 3 degree-of-freedom point-masses constrained in their motion. Collectively, these kinematic constraints capture the incompressibility of the fluid and couple the relative motion of the SPH markers. Notably, the effect of these constraints comes into the momentum equations as well. Indeed, the Lagrange multiplier forcing term that arises from the compressibility constraint acts as the pressure gradient term in the momentum balance equations \cite{espanol_smoothed_2003,constrFluid2007,claudeIncompFluids2012,hammadConstrFluid2018}. 

The cornerstones of WCSPH and ISPH were, respectively, the use of a stiff state equation for recovering the pressure, and the use of a Poisson equation to produce a pressure field that enforces incompressibility. In KCSPH, the cornerstone is the use of holonomic kinematic constraints to enforce incompressibility; i.e., one demands that at the location of marker $i$ the density assumes a reference value $\rho_0$:
\begin{equation} \label{eq:CF_constraint}
C^f_i = \frac{\rho_i-\rho_0}{\rho_0} = 0\;.
\end{equation}
\noindent If the time-derivative of the constraints in Eq.~\ref{eq:CF_constraint} is satisfied at the velocity level, i.e.  $\dot{C}^f_i=d\rho/dt=0$, the following will emerge after applying Eq.~\ref{eq:SPH_f}; i.e., invoking the SPH machinery:
\begin{align} \label{eq:CF_constraint_dot}
\dot{C}^f_i =  \frac{d}{dt}(\frac{\rho_i}{\rho_0} -1) \approx &  \sum_j \frac{m_j}{\rho_0} \frac{dW_{ij}}{dt} =  \sum_j \frac{m_j}{\rho_0} \frac{d W_{ij}}{d \vect{x}_{ij}}. \frac{d \vect{x}_{ij}}{dt} \nonumber\\
&=\sum_j \frac{m_j}{\rho_0} \nabla_i W_{ij}. (\textbf{u}_i-\textbf{u}_j) = - \sum_j \frac{m_j}{\rho_0} \nabla_i W_{ij}. (\textbf{u}_j-\textbf{u}_i) \; .
\end{align}
\noindent
Above, if $\rho_j\approx \rho_0$, the last term mimics $- <\nabla. \textbf{u}>_i$, see Eq.~\ref{eq:Standard_G}. Elements in row $i$ of the constraint Jacobian matrix $\nabla_{\bf q} {\bf g}({\bf q},t)\equiv {\bf G}$ (see Eq.~\ref{eq:Newton_Euler}) are associated with the constraint $ g_i=\dot{C}^f_i=0$. These entries in ${\bf G}$ may be obtained via Eq.~\ref{eq:CF_constraint_dot} as follows:
\begin{align} \label{eq:CF_constraint_Jac}
&\dot{C}^f_i =  \sum_j \frac{m_j}{\rho_0} \nabla_i W_{ij} \textbf{u}_i - \sum_j \frac{m_j}{\rho_0} \nabla_i W_{ij} \textbf{u}_j \; \nonumber\\
& \qquad \Rightarrow \qquad
{G}_{ii} = \frac{1}{\rho_0}\sum_{k \ne i} m_k \nabla W_{ik} \quad \mbox{and} \quad 
{G}_{ij} =-\frac{m_j}{\rho_0}\nabla_i W_{ij}.
\end{align}
\noindent
Each density constraint on a marker contributes to a single row in the full Jacobian matrix which has $3\nParticles$ columns. This matrix is sparse and its rows have three values at the columns corresponding to the current marker $i$ and three values for each marker $j$ within the support domain of $i$. More specifically, for row $i$
\begin{align}\renewcommand{\arraystretch}{1.2}
\mathbf{G}_i=& \begin{bmatrix}
\cdots, & 
\smash[b]{\underbrace{\begin{matrix}\frac{1}{\rho_0}\sum_{j \in \support{i}} m_k \nabla_i W_{ik}^T \end{matrix}}_{i^{th}\text{ element} }} ,
&\cdots, &\smash[b]{\underbrace{\begin{matrix}-\frac{m_j}{\rho_0}\nabla_i W_{ij}^T\end{matrix}}_{j^{th}\text{ element s.t. }{j \in \support{i}} }}, & \cdots
\end{bmatrix}_{1\times 3\nParticles} \; . \label{eq:DMat}
\end{align}\\
It is informative to compare Eqs.~\ref{eq:DMat} and \ref{eq:AGMat} to underline similarities between the constraint Jacobian row  $\mathbf{G}_i$ and the discretized gradient operator $\mathbf{A}_i^G$. Qualitatively, the density constraint Jacobian matrix is similar to the discretized gradient operator -- satisfying  $\mathbf{G} \mathbf{u}=0$ is the analog of  $\nabla.\mathbf{u}=\mathbf{A}^{Gx} \mathbf{u}_x+\mathbf{A}^{Gy} \mathbf{u}_y+\mathbf{A}^{Gz}\mathbf{u}_z=0$ (see Eq.~\ref{eq:divergence_disc}). This connection between $\mathbf{G}$ and $\mathbf{A}^G$ comes further into play when one considers how the pressure factors into the Newton-Euler equations of motion. Imposing the kinematic constraint $\dot{C}^f=\mathbf{G} \mathbf{\dot{q}}=0$ leads to the presence of a Lagrange multiplier. The multipliers in Eq.~\ref{eq:Newton_Euler} play a role similar to that of the pressures. This becomes clearer when the force term associated with the Lagrange multipliers in Eq.~\ref{eq:Newton_Euler}, i.e. $\big(\nabla_{\bf q} {\bf g}({\bf q},t)\big)^T \bf{\Lambda}$ is written as $\mathbf{G}^T \bf{\Lambda}$, where the connection between the $\mathbf{G}$ matrix and the discretized gradient matrix $\mathbf{A}^G$ is considered. It follows that KCSPH replaces the pressure gradient term $-\frac{1}{\rho} \mathbf{A}^{G}\mathbf{p}$ in the space-discretized Navier-Stokes (Eq.~\ref{eq:NS_discretized}) with $\mathbf{G}^T \bf{\Lambda}$. Considering the resemblance of $\mathbf{A}^{G}$ and the $-\mathbf{G}$ matrices, we may conclude that the Lagrange multipliers scaled (element-wise)  by the particles' volume (${\bf\Lambda}/V$) is the mechanical pressure $\mathbf{p}$ in the Navier-Stokes equations. 

Up to this point, the KCSPH discussion focused exclusively on obtaining the SPH equations of motion in  the form of Section \ref{eq:Newton_Euler}'s differential algebraic equations. In other words, the spatial discretization step has been accomplished. How to discretize and solve in time these differential algebraic equations goes beyond the scope of this manuscript and the interested reader is referred to \cite{hammadConstrFluid2018}.

\subsection{Boundary Conditions}
\label{sec:BC}
%Imposing boundary conditions is relatively more challenging in Lagrangian methods such as SPH in comparison with Eulerian methods such as Finite Difference and Finite Volume. Many researchers came up with new techniques for boundary condition enforcement in SPH. 
Herein, imposing boundary conditions (BCs) falls back on the use of so-called {\textit{boundary conditions enforcing}} (BCE) markers. These are fictitious markers, rigidly attached to the boundary in a buffer zone that runs several layers of BCE markers deep. These markers are used to enforce the no-slip and no-penetration conditions; i.e., to satisfy the $\mathbf{u}^*= \mathbf{0}$ condition on the boundary. This condition is implemented differently in WCSPH, ISPH and KCSPH.


In one widely used approach \cite{Adami2012}, the \textit{expected} kinematic attributes of the markers, calculated from the motion of the solid phase at the location occupied by the markers, are different from their \textit{assigned} values. The latter are calculated such that the no-slip and no-penetration boundary conditions are implicitly enforced at the fluid-solid interface. The no-slip condition states that the velocity of the BCE markers should oppose the velocity of the fluid particles such that the average \textit{relative} fluid-solid velocity at the interface is zero; i.e., the average velocity at the interface is the \textit{expected} interface velocity. Herein, the \textit{induced} velocity $ \tilde{\mathbf{u}}_a$ at the position of marker $a$ is computed from the velocity of the fluid markers as
\begin{equation}\label{eq:BCE_extrapolation}
\tilde{\mathbf{u}}_a = \frac{\sum\limits_{b \in \mathbf{F}} {\mathbf{u}_b W_{ab}}}{\sum\limits_{b \in \mathbf{F}} {W_{ab}}},
\end{equation}
where $\mathbf{F}$ denotes a set of fluid markers that are within the compact support of the BCE marker $a$. The no-slip condition holds if $(\tilde{\mathbf{u}}_a +\mathbf{u}_a)/2=\mathbf{u}_a^p$; in other words the \textit{assigned} velocity of marker $a$ is \cite{Adami2012}:

\begin{equation} \label{eq:vBCE_Adami}
\mathbf{u}_a = 2 \mathbf{u}^{p}_a - \tilde{\mathbf{u}}_a \; ,
\end{equation}
where $\mathbf{u}^{p}_a$ is the \textit{expected} wall velocity at the position of the marker $a$, and $\tilde{\mathbf{u}}_a$ is an extrapolation of the smoothed velocity field of the fluid phase to the BCE markers.


%In one widely used approach \cite{Adami2012}, the \textit{expected} velocity of the BCE markers is dictated by the motion of the boundary/solid to which they belong and the BCE marker kinematics is calculated such that the no-slip and no-penetration boundary conditions are implicitly enforced at the interface. The \textit{assigned} velocity, $\mathbf{u}_a$, of a BCE marker $a$ is computed as {\SBELcomment{i'm a bit confused by this; the marker for which $\mathbf{u}_a$ is set below, is this a fluid marker of a BCE marker? In my mind, the eq. below pertains to *fluid* markers that have particle deficiency. Am i right?}} \cite{Adami2012}


The pressure of a BCE marker may be calculated via a force balance condition at the wall interface, which leads to \cite{Adami2012}
\begin{equation} \label{eq:pBCE_Adami}
p_a = \frac{\sum\limits_{b \in \bf F} {p_b W_{ab}} + \left( \mathbf{g} - \mathbf{a}_a^p \right) \cdot \sum\limits_{b \in \bf F} {\rho_b \mathbf{r}_{ab} W_{ab} }}{\sum\limits_{b \in \bf F} {W_{ab}}},
\end{equation}
where $\mathbf{g}$ is the gravitational acceleration and $\mathbf{a}_a^p$ is the acceleration of the boundary/solid at the location of BCE marker $a$.

This approach is fairly straightforward to implement in WCSPH -- one can modify the velocity and  pressure of the BCE markers according to Eqs.~\ref{eq:vBCE_Adami} and \ref{eq:pBCE_Adami}, respectively. In regard to the ISPH method, ($i$) the no-slip boundary condition is implemented in the linear system of Eq.~\ref{eq:predict_disc}; and, ($ii$) pressure boundary conditions should be incorporated into the linear system in Eq.~\ref{eq:correct_disc}. In regard to ($i$), the modified row of the linear system associated with the boundary marker $a$ for the velocity equation Eq.~\ref{eq:predict_disc} is
\begin{align}\renewcommand{\arraystretch}{1.2}
&\mathbf{A}^{v}_a \mathbf{u}_x=  2 ({u}_x^{p})_a {\sum\limits_{b \in \mathbf{F}} {W_{ab}}},
\quad
\mathbf{A}^{v}_a \mathbf{u}_y=  2 ({u}_y^{p})_a {\sum\limits_{b \in \mathbf{F}} {W_{ab}}}, 
\quad
\mathbf{A}^{v}_a \mathbf{u}_z=  2 ({u}_z^{p})_a {\sum\limits_{b \in \mathbf{F}} {W_{ab}}},  \\
&\mathbf{A}^{v}_a= \begin{bmatrix}
\cdots, & 
\smash[b]{\underbrace{\begin{matrix}{\sum\limits_{b \in \mathbf{F}} {W_{ab}}} \end{matrix}}_{a^{th}\text{ element}}} ,
&\cdots, &\smash[b]{\underbrace{\begin{matrix} {W_{ab}}\end{matrix}}_{b^{th}\text{ element s.t. }{b \in \mathbf{F} \text{ and } \in \support{a}} }}, & \cdots
\end{bmatrix} \label{eq:U_BC}.
\end{align}
\\
In regard to ($ii$), the interplay between Eqs.~\ref{eq:press_disc} and \ref{eq:pBCE_Adami} leads to
\begin{align}\renewcommand{\arraystretch}{1.2}
&\mathbf{A}^{p}_a \mathbf{p}= \left( \mathbf{g} - \mathbf{a}_a^p \right) \cdot \sum\limits_{b \in \bf F} {\rho_b \mathbf{r}_{ab} W_{ab} },\\
&\mathbf{A}^{p}_a= \begin{bmatrix}
\cdots, & 
\smash[b]{\underbrace{\begin{matrix}{\sum\limits_{b \in \mathbf{F}} {W_{ab}}} \end{matrix}}_{a^{th}\text{ element}}} ,
&\cdots, &\smash[b]{\underbrace{\begin{matrix} {-W_{ab}}\end{matrix}}_{b^{th}\text{ element s.t. }{b \in \mathbf{F} \text{ and } \in \support{a}} }}, & \cdots
\end{bmatrix} \label{eq:p_BC}.
\end{align}
\\

Note that instead of Eq.~\ref{eq:pBCE_Adami}, one can enforce at the boundary the condition that the pressure satisfy $\nabla p^{n+1}\;.\;\mathbf{n}|_{\partial \Omega}=0$, which mimics the traditional Eulerian handling of pressure boundary conditions. Insofar as the discretized pressure equation is then concerned, the row in the linear system in Eq.~\ref{eq:press_disc} that corresponds to boundary particle $a$ will read
\begin{align*}\renewcommand{\arraystretch}{1.2}
&\mathbf{A}^{p}_a \mathbf{p}=  0  \\
&\mathbf{A}^{p}_a= \begin{bmatrix}
\cdots, & 
\smash[b]{\underbrace{\begin{matrix}{\sum\limits_{b \in \mathbf{F}} \mathbf{A}^G_{ab} \cdot \mathbf{n}_a } \end{matrix}}_{a^{th}\text{ element}}} ,
&\cdots, &\smash[b]{\underbrace{\begin{matrix} \mathbf{A}^G_{ab} \cdot \mathbf{n}_a \end{matrix}}_{b^{th}\text{ element s.t. }{b \in \mathbf{F} \text{ and } \in \support{a}} }}, & \cdots
\end{bmatrix}.
\end{align*}\\
Above, $\mathbf{A}^G_{ab} \in \mathcal{R}^3$ is the $b^{th}$ element of discretized gradient matrix $\mathbf{A}^G_a$ (see Eq.~\ref{eq:AGMat}), $\mathbf{n}_a$ is the surface normal vector at the position of particle $a$, and $\mathbf{p}$ was defined in Eq.~\ref{eq:pres_vec}. 


Finally, KCSPH handles boundary conditions via multi-body dynamics techniques \cite{hammadConstrFluid2018}. The no-penetration condition between SPH markers and boundaries is enforced just as non-penetrating contacts are treated in multi-body dynamics; i.e., via unilateral constraints (see Eq.~\ref{eq:Newton_Euler}). This strategy allows for $(i)$ strict (per marker) enforcement of the no-penetration, and $(ii)$ more flexibility in terms of the geometry of the boundary. On the downside, this strategy can only be used to enforce the \textit{no-penetration} condition but not the \textit{no-slip}, which is unlike WCSPH and ISPH.

\subsection{Viscoplasticity}




\section{Time Integration of Discrete Systems}




By taking one time-derivative of the constraint equations, the velocity-level constraint equations which must also be satisfied are obtained as $\nabla_q \vect{g}_i^T \matr{T}\left(\vect{q}\right)\vect{v}+\frac{\partial\vect{g}_i}{\partial t}=0$ \cite{TasoraAnitescuCMAME10}.




\section{Optimization Formulation of Discrete Systems}
\section{Lagrangian-Lagrangian Fluid-Solid Interaction}
