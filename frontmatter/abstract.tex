%!TEX root = ../dissertation.tex
This thesis investigates computational methods pertaining to multi-physics dynamics problems featuring continua, discrete systems, and their coupling. Specifically, computational methods for solving governing equations of fluids, solids, and their interaction are studied in a partitioned Lagrangian and parallel computing framework. \\
%
For fluid dynamics problems, the focus was on the use of Smoothed Particle Hydrodynamics (SPH) as a Lagrangian discretization method for modeling and simulation of fluid flows. More specifically, the Navier-Stokes equations are solved via an implicit-in-velocity and pressure algorithm using a Chorin-style splitting technique for both Newtonian and non-Newtonian fluid models. This implicit time integration allows for large time-steps while simulating a broad spectrum of fluid flows ranging from highly viscous to flows with moderately large Reynolds numbers in the laminar regime. The same continuum approach is adopted to resolve the dynamics of discrete systems such as granular media by choosing a proper constitutive equation for the deviatoric part of the stress tensor. The equivalence of the granular constitutive equation to the Herschel-Bulkley fluid model allows for treating the granular material in a continuum sense as a non-Newtonian fluid. A bi-viscosity model is employed to numerically handle the yield stress within the Navier-Stokes framework.\\
%
For solid mechanics, the rigid-body dynamics governing equations account for frictional contacts via a differential variational equality method along with an implicit time integration scheme allowing for large time-steps. The uniqueness of the optimization problem arising from frictional contacts is investigated and the Tikhonov regularization technique is used to select the minimum norm solution. When bodies are flexible/compliant, their dynamics is captured via the absolute nodal coordinate formulation, a non-linear finite element method designed to handle simultaneously large deformations and large displacements/rotations.\\
%
Lastly, the two-way, dynamics coupling of the fluid and solid phases is done explicitly in a partitioned framework. So-called Boundary Condition Enforcing (BCE) markers capturing the motion of the solid phase are employed to impose no-slip and impenetrability conditions for the fluid phase. Subsequently, the hydrodynamics forces are transferred to the rigid and flexible multi-body dynamics systems as external forces.\\
%
The high computational load associated with the simulation of fluid-solid interaction problems typically leads to long compute times. To address this issue, the software solution developed under this work relies on: high-performance computing on graphics processing unit cards to parallelize the fluid solver; and multi-core parallel computing and vectorization to accelerate the rigid/flex-body dynamics solver. The software implementation of all algorithms discussed in this work is publicly available on GitHub in an open-source C++ software package, called Chrono, which is released under a permissive BSD3 license.
